\documentclass[preview]{standalone}

\usepackage[english]{babel}
\usepackage{amsmath}
\usepackage{amssymb}

\begin{document}

\begin{center}
(OBMEP 2016N1Q6) Joãozinho pinta anéis encaixados, 
 cada um deles dividido em seis partes iguais. 

No primeiro anel (o menor deles) Joãozinho pinta de cinza algumas partes, à sua escolha. 

Do segundo anel em diante, ele pinta de cinza 
 somente as partes em contato com duas partes de cores diferentes do anel anterior. 

Observe um exemplo:
\end{center}

\end{document}

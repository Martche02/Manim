\documentclass[preview]{standalone}

\usepackage[english]{babel}
\usepackage{amsmath}
\usepackage{amssymb}

\begin{document}

\begin{center}
C) Explique por que, independentemente de como Joãozinho pintar o primeiro anel, os demais anéis sempre terão uma
quantidade par de partes pintadas de cinza.
\end{center}

\end{document}

\documentclass[preview]{standalone}

\usepackage[english]{babel}
\usepackage{amsmath}
\usepackage{amssymb}

\begin{document}

\begin{flushleft}
(OBMEP 2021N3Q6 ADAPTADA) São dispostas 10 moedas em um círculo.
Inicialmente, todas as dez moedas são colocadas com a face coroa voltada para cima e um
ponteiro aponta para a posição A. 


 Esse ponteiro começa a se movimentar no sentido anti-horário,
saltando de uma posição para a outra mais próxima. 
Após cada salto,

      • Se o ponteiro apontar para uma moeda
com a face cara para cima, nada acontece;

      • Se o ponteiro apontar para uma moeda
com a face coroa para cima, deve-se, então,
virar a moeda seguinte.
\end{flushleft}

\end{document}
